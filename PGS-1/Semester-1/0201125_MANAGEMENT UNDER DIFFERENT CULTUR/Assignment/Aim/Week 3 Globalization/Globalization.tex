\documentclass[11pt]{article}

\usepackage{xcolor}
\usepackage{scrlayer-scrpage}

\newcommand{\MyName}{P. Lerdkijrachapong}

\usepackage{blindtext}

\clearscrheadfoot
\rohead{\MyName}
\lehead{\today}

\chead{TikTok}
\lohead{21 August 2025}
\rehead{\MyName}    
\cefoot{\thepage}

\begin{document}
Amidst the rise of new age of information technology, came along various social media platform. From the starting of video sharing in YouTube, to short media in Vine; or photo sharing in Facebook, which evolves into Instagram. One platform excel into the pinnacle of modern day informative technology\textemdash \textbf{TikTok}. 

In 2016, TikTok started as a application in China, named Douyin, This application rapidly gain popularity among Chinese, and Thai people. Consequently, in 2017, ByteDance, the company, developed the ``international" version of Douyin, hence, globalising Douyin into TikTok. The application quickly became one of the most populate, and most influential social media application within a few year of going global, re\textendash igniting  the popularity of short video. However, with the rise of its broader amount of user\textemdash including the substantial amount of young and underage user\textemdash the rise of lack of regulation began to hit surface. Within a small amount of time, this regulation has been quickly patched, and evolved into what we're agreeing as \textit{Community Guideline}; little did we know, this guideline will affect the world more than we think.

For the first time, the evolution of linguistic has been affect by the mean other than the passing of generation. Moreover, this evolution happens faster than ever. For example, most of the word English language usually evolved from West Germanic Group, or Latin; which the evolution gradually happens in the span of centuries. However, with the rise of social media in whole, and especially the rise of TikTok, came the rapid evolution of the language. For example, TikTok's regulation of the permitted word in the application leads to the changes in how the new generation talks.

The community guideline strictly prohibited the usage of sensitive content associated with death (including "dead", "kill", or "suicide"). This regulation changed the way that TikTok's user communicate in this topic. For example, instead of ``\textbf{Death}", the user use the word ``\textbf{Un\textendash Alive}". Although the older generation used to seeing or communicating with the word ``Death", younger generation has associate the concept of death, to the word ``Un\textendash Alive", not because the fear of the concept of death, but due to how they developed their linguistic, and personality through social media.

This example is only one of many consequences of the rise of new information technology; and one of the effect of how globalisation began to reform our world, and new generation.
\end{document}
