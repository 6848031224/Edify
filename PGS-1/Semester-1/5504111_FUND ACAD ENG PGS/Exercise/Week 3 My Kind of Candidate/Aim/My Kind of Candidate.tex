\documentclass[11pt]{article}

\usepackage{xcolor}
\usepackage{scrlayer-scrpage}

\newcommand{\MyName}{P. Lerdkijrachapong}

\usepackage{blindtext}

\clearscrheadfoot
\rohead{\MyName}
\lehead{\today}

\chead{My Kind of Candidate}
\lohead{19 August 2025}
\rehead{\MyName}    
\cefoot{\thepage}


\begin{document}
When we talk about a states leader, we often sought for the perfect candidate\textemdash however\textemdash is there really is one? The ideal ``perfect" candidate varies due to the political perspective of voters, or persons. In the other word, our political alignment affects who, and what we consider as perfect candidate. However, amongst those mist of political view, we certainly agree on few qualities. \textbf{One}, the leader should understand the value of their nation, and. \textbf{Two}\textemdash nation before international\textemdash while nation's leader may differs in political perspective, the concept of ``doing the best for the nation" is what ultimately keeps the nation alive.

Contrary to the pack of wolves, or the herd of elephants; nation isn't just a group of people; but; nation is the accumulate of decades and centuries of wisdom, knowledge, and tradition; \textbf{the pinnacle of human's social instinct}.

To illustrate, Thailand's Songkran festival has dual origins: it traces back to an ancient Hindu festival in India, specifically Makar Sankranti, which marks the solar new year and involves water-related rituals. The Thai word "Songkran" itself is derived from the Sanskrit term ``saṅkrānti," meaning the astrological passage of the sun into a new zodiac sign. This suggests a significant Indian influence on the festival's timing and name; Songkran tradition also ties to Buddhist folk myths, and agricultural seasonal changes in Thailand, with influences from Hindu astrology and Theravada Buddhist practices that spread into Southeast Asia over centuries. 

As previously exemplified, a nation's, in this case, Thailand's, tradition doesn't quickly emerge, but a long\textemdash term accumulation of generational knowledges, and practices. At last, nation without it's deeply\textendash rooted tradition, generational\textendash marinated culture, or centuries\textendash old knowledge, loses its beauty; and the  good leader should be able to understand nation's value, and make decision according to what's the best for them. For me, the ability to preserve what makes nation, a nation, while leading the nation toward the future, is the quality that ``perfect" candidate should have.


\end{document}
