\documentclass[aspectratio=169]{beamer}

%========================
% Theme & Color
%========================
\usetheme{Berlin}
\usecolortheme{whale}

%========================
% Typography & Layout Tweaks
%========================
\setbeamertemplate{navigation symbols}{}                % remove nav symbols
\setbeamertemplate{frametitle continuation}[from second]
\setbeamerfont{frametitle}{size=\Large,series=\bfseries}
\setbeamerfont{framesubtitle}{size=\small}
\setbeamerfont{block title}{size=\normalsize,series=\bfseries}
\setbeamerfont{block body}{size=\small}
\setbeamerfont{itemize/enumerate body}{size=\small}
\setlength{\parindent}{0pt}
\setlength{\parskip}{0.5em}

%========================
% Metadata
%========================
\title[Eurocentrism \& Great Divergence]{From Eurocentrism to the Great Divergence:\\Study Group}
\author[P.\,Lerdkijrachapong]{Punnawat Lerdkijrachapong}
\institute[Study Group]{Study Group 2}
\date{\today}

\begin{document}

%========================
% Title Slide
%========================
\begin{frame}
  \titlepage
\end{frame}

%========================
% Overview
%========================
\begin{frame}{Overview}
  This presentation unfolds in three parts:
  \begin{enumerate}[<+->]
    \item Introduction to Eurocentrism
    \item The Great Divergence in Economic History
    \item Key Takeaway \& Conclusions
  \end{enumerate}
\end{frame}

%========================
% Section: Eurocentrism
%========================
\section{Eurocentrism}

\begin{frame}{What Is Eurocentrism?}
  \begin{block}{Definition}
    A worldview that positions Western European experience as the universal norm, marginalizing other histories, cultures, and ideas.
  \end{block}
  \vspace{0.5em}
  \begin{itemize}[<+->]
    \item Shapes academic disciplines, politics, and education  
    \item Frames non-European societies as deviations from a European baseline  
  \end{itemize}
\end{frame}

\begin{frame}{Historical-Contextual Eurocentrism}
  Credits Europe with unique rationality and progress based on Greco-Roman heritage.
  \vspace{0.5em}
  \begin{block}{Key Features}
    \begin{itemize}[<+->]
      \item Emphasis on European “exceptionalism” in governance and innovation  
      \item Selective recounting of the Middle Ages and Renaissance  
    \end{itemize}
  \end{block}
\end{frame}

\begin{frame}{Ideological \& Residual Eurocentrism}
  \begin{block}{Ideological}
    Promotes European norms and values as the pinnacle of human development.
  \end{block}
  \vspace{0.5em}
  \begin{block}{Residual}
    Subtle persistence of Eurocentric assumptions in modern policies and theories.
  \end{block}
  \vspace{0.5em}
  \begin{itemize}[<+->]
    \item “White Man’s Burden” religious-missionary justification  
    \item IR theories built on European sovereignty models  
  \end{itemize}
\end{frame}

\begin{frame}{Philosophical Eurocentrism}
  \begin{block}{Deep Claim}
    European reason provides the sole universal foundations; non-European philosophies become ethnographic curiosities.
  \end{block}
  \vspace{0.5em}
  This prompts a rethink of our reading lists and curricula.
\end{frame}

%========================
% Section: Great Divergence
%========================
\section{The Great Divergence}

\begin{frame}{Part II: The Great Divergence}
	The "Great Divergence" refers to the widening gap in economic prosperity, technology, and
living standards that emerged between the Western world and much of Asia, Africa, and
Latin America from the late 18th century onwards—a process deeply linked to colonialism,
industrialization, evolving models of economic growth, and the long shadows they cast on
global inequality.
\end{frame}

\begin{frame}
	The Great Divergence is widely accepted as having unfolded between 1750 and
1900, reshaping the world.

	\begin{block}{Key Fact}
		\begin{itemize}[<+->]
			\item In 1500, differences in standards of living were modest (factor of 4); today, the ratio
					between richest and poorest nations exceeds 40:1.
			\item Core Drivers:
				\begin{itemize}
					\item The Industrial Revolution (Western Europe, North America)
					\item Colonial expansion and resource transfer (from colonies to metropole)
					\item Institutional change (property rights, inclusive vs. extractive institutions)
					\item Technology adoption and innovation
				\end{itemize}
		\end{itemize}
	\end{block}

\end{frame}

\begin{frame}{Part II: The Great Divergence}
  How did Western Europe surge ahead in wealth and technology between 1750 and 1900?
  We explore population dynamics, market forces, innovation, and colonial power.
\end{frame}

\begin{frame}{Malthusian Model: Stagnation by Design}
	Malthusian Model, named after Thomas Robert Malthus's 1798 work, is
foundational to understandingpre-industrial economic
  \begin{block}{Core Mechanism}
    Income gains trigger population growth until wages return to subsistence.
  \end{block}
  \vspace{0.5em}
  \begin{itemize}[<+->]
    \item Income \(\uparrow\) → Birth rates \(\uparrow\), Mortality \(\downarrow\)  
    \item Population \(\uparrow\) → Income per capita \(\downarrow\)  
    \item Equilibrium at subsistence wage  
  \end{itemize}
\end{frame}

\begin{frame}{Smithian Growth: Division of Labour}
	Unlike Malthus, Smith believed that productivity could rise with
	scale, specialization, and institutional development:
  \begin{block}{Virtuous Cycle}
    \begin{enumerate}[<+->]
      \item Capital accumulation → Higher output  
      \item Higher output → Larger market  
      \item Larger market → Greater specialization  
      \item Specialization → New machinery and tools  
    \end{enumerate}
  \end{block}
  Institutional context: Property rights and stability amplify this cycle.
\end{frame}

\begin{frame}{Solow Model: Technology as Prime Mover}
	The 20th-century Solow model addresses the limitations of both Malthus and
	Smith by centering sustained technological progress and relaxing land constraints
  \begin{theorem}
    If technological progress exceeds population growth, steady-state per-capita income rises indefinitely.
  \end{theorem}
  \vspace{0.5em}
  \begin{itemize}[<+->]
    \item Escapes Malthusian land constraints  
    \item Tech shifts production function upward  
  \end{itemize}
\end{frame}

\begin{frame}{Endogenous Growth: Innovation \& Spillovers}
  Ideas are non-rival public goods, generating increasing returns at the aggregate level.
  \vspace{0.5em}
  \begin{block}{Implications}
    \begin{itemize}[<+->]
      \item Human capital and R\&D become central to growth  
      \item Knowledge spillovers create social returns beyond private gains  
    \end{itemize}
  \end{block}
\end{frame}

\begin{frame}{Industrial Revolution: Regional Case Studies}
	The Industrial Revolution, starting roughly in late-18th-century Britain, is the locus
	classicus of the Great Divergence
  \begin{block}{Britain’s Lead}
    Mechanized textiles and steam power boosted GDP per capita from \$1,706 (1820) to \$3,190 (1870).
  \end{block}
  \vspace{0.5em}
  \begin{itemize}[<+->]
    \item UK: \$1,706 $\to$ \$3,190  
    \item India: \$533 $\to$ \$533  
    \item China: \$600 $\to$ \$530  
    \item Japan (post-1868): \$669 $\to$ \$737 
  \end{itemize}
\end{frame}

\begin{frame}{Colonialism’s Economic Impact and Institutional Legacies}
	Colonialism reshaped global economics by:
	\begin{block}{Heterogeneous Effects}
			\begin{itemize}
				\item Reorienting trade, production, and political institutions for the benefit of the
metropole
				\item Establishing \textbf{extractive institutions} (favoring raw material exports, limiting local
entrepreneurial growth) in densely populated or malaria-prone regions
				\item Creating \textbf{inclusive institutions} (with property rights, broader suffrage) in settler
colonies with mortal climates
			\end{itemize}
	\end{block}

\end{frame}

\begin{frame}{Colonialism: Institutions \& Inequality}
  \begin{block}{Extractive vs.\ Inclusive}
    \begin{itemize}[<+->]
      \item Extractive (Congo, India): Authoritarian institutions, persistent poverty  
      \item Settler (US, Canada): Inclusive institutions, high prosperity  
    \end{itemize}
  \end{block}
\end{frame}

\begin{frame}{The Racial \& Ideological Roots of Empire}
  \begin{exampleblock}{Racial Ideologies}
    Europeans claimed innate superiority, justifying segregation and forced labor.
  \end{exampleblock}
  \vspace{0.5em}
  \begin{alertblock}{Enduring Impact}
    These hierarchies shaped legal systems and collective memory, persisting in institutional biases today.
  \end{alertblock}
\end{frame}

\begin{frame}{Colonial Education \& Its Aftermath}
  \begin{block}{Colonial Model}
    A small elite received Eurocentric curricula to serve colonial administrations.
  \end{block}
  \vspace{0.5em}
  Post-independence reforms expanded access unevenly, leaving skill gaps and cultural tensions.
\end{frame}

\begin{frame}{Globalization \& Persistent Inequality}
  \begin{block}{Colonial Legacies in Trade}
    Many former colonies remain commodity exporters in ``trade traps," (exporting raw materials or low-value-added goods) while high-value industries concentrate in the Global North.
  \end{block}
  \vspace{0.5em}
\end{frame}

\begin{frame}{Globalization \& Persistent Inequality}
  \begin{block}{Colonial Legacies in Culture}
   Colonies continue to be marked by hierarchies established in the
colonial period, with Western languages, institutions, and norms dominating
international life.  \end{block}
  \vspace{0.5em}
\end{frame}

\begin{frame}{Globalization \& Persistent Inequality}
  \begin{block}{Colonial Legacies in Inequality}
   The North-South divide, shaped by centuries of extraction and
institutional disparity, is now reinforced by technological gaps, capital flows, and
trade structures favoring wealthy economies.  \end{block}
  \vspace{0.5em}
\end{frame}

%========================
% Conclusions
%========================
\begin{frame}{Key Takeaway \& Conclusions}
  \begin{itemize}[<+->]
    \item \textbf{Historical growth models} (Malthusian, Smithian, Solow, Romer) provide necessary
lenses for interpreting global economic trajectories and understanding why some
societies broke free of stagnation while others remained mired in poverty.  
    \item \textbf{Colonial expansion and deindustrialization} created lasting disparities in
institutions, education systems, and trade structures, fostering persistent gaps in
income and opportunity. 
    \item \textbf{Technological change} is indispensable for development but must be harnessed
inclusively to avoid sharpening divides. 
    \item \textbf{Racial and ideological legacies} shape not only policies of the past but the structures
of present-day societies—from who controls capital and land, to who enjoys
educational and political rights.
	\item \textbf{Globalization} connects, but also divides—reflecting colonial legacies in trade,
education, and culture.
  \end{itemize}
  \vspace{0.5em}
\end{frame}

%========================
% Thank You
%========================
\begin{frame}{Thank You}
  Questions and discussion are not welcome!  
  \vspace{1em}
  Contact: \texttt{punnawatcont@gmail.com}
\end{frame}

\end{document}
