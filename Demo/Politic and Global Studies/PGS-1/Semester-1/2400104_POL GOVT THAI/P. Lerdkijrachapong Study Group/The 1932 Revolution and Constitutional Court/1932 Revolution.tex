\documentclass[aspectratio=169]{beamer}
\usepackage{xcolor}
%========================
% Theme & Color
%========================
\usetheme{Berlin}
\usecolortheme{whale}

%========================
% Typography & Layout Tweaks
%========================
\setbeamertemplate{navigation symbols}{}                % remove nav symbols
\setbeamertemplate{frametitle continuation}[from second]
\setbeamerfont{frametitle}{size=\Large,series=\bfseries}
\setbeamerfont{framesubtitle}{size=\small}
\setbeamerfont{block title}{size=\normalsize,series=\bfseries}
\setbeamerfont{block body}{size=\small}
\setbeamerfont{itemize/enumerate body}{size=\small}
\setlength{\parindent}{0pt}
\setlength{\parskip}{0.5em}

%========================
% Metadata
%========================
\title[The 1932 Revolution as a Critical Juncture]{The 1932 Revolution as a Critical Juncture}
\author[P.\,Lerdkijrachapong \& C. Bangwatana]{Punnawat Lerdkijrachapong \& Chalongraj Bangwatana}
\institute[Study Group]{Study Group 2}
\date{\today}

\begin{document}

%========================
% Title Slide
%========================
\begin{frame}
  \titlepage
\end{frame}

%========================
% Overview
%========================
\begin{frame}{Overview}
  This presentation unfolds in three parts:
  \begin{enumerate}[<+->]
  	\item Functions of States
    \item Khana Ratsadon 1932 (2475)
    \item Constitutional Court
  \end{enumerate}
\end{frame}

%========================
% Section: Functions of States
%========================
\section{Functions of States}

\begin{frame}{Part I: Functions of States}
	\begin{block}{How Many Functions?}
		\begin{enumerate}
			\item Primary
			\item Secondary
		\end{enumerate}
	\end{block}
\end{frame}

\begin{frame}{Primary Function}
Examples:
		\begin{itemize}
			\item Security and Protection
			\item Regulation and Law
			\item Dispute Resolution
			\item Economic
		\end{itemize}
\end{frame}

\begin{frame}{Secondary Function}
Examples:
	\begin{itemize}
		\item Social Welfare
		\item Diplomat
		\item Provision of Public Utilities
	\end{itemize}
\end{frame}
%========================
% Section: Khana Ratsadon 1932 (2475)
%========================
\section{Khana Ratsadon ``The People’s Party"}

\begin{frame}{Part II: Khana Ratsadon}
  \begin{block}{What Is Khana Ratsadon?}
   Khana Ratsadon ``The People’s Party", is a group of elite people who studied in France and Switzerland and wished to
change of regime (absolute monarchy \rightarrow{} constitutional monarchy)
  \end{block}
\end{frame}

\begin{frame}{Members of Khana Ratsadon}
	\begin{enumerate}
		\item Pridi Banomyong
		\item Prayun Phamonmontri
		\item Plaek Phibunsongkhram
		\item Tassanai Mitrpakdi
		\item Tua Laphanukorn
		\item Jaroon Singhaseni
		\item Naep Phahonyonthin
	\end{enumerate}
\end{frame}

\begin{frame}{Members of Khana Ratsadon}
	\begin{enumerate}
		\item Pridi Banomyong \colorbox{pink}{(Former Thai Prime Minister)}
		\item Prayun Phamonmontri
		\item Plaek Phibunsongkhram \colorbox{pink}{(Former Thai Prime Minister)}
		\item Tassanai Mitrpakdi
		\item Tua Laphanukorn
		\item Jaroon Singhaseni
		\item Naep Phahonyonthin
	\end{enumerate}
\end{frame}

\begin{frame}{Principles of Khana Ratsadon}
		\begin{itemize}[<+->]
			\item To maintain the supreme power of the Thai people such as independency of Thai politics, jurisdiction, economy, and etc.
			\item To maintain national security, in order to minimise wrong doing.
			\item To maintain the economic welfare of the Thai people in accordance with the National Economic Project, where the new regime must provide job opportunity to every citizen, and lay down national economic project, and must not let people starve.
			\item To protect the equality of the Thai people. (And not just allowing the royal to have more rights than commoners as it is the current situation at the time.)
			\item To maintain the people's rights and liberties, insofar as they are not inconsistent with any of the above mentioned principles.
			\item To provide public education for all citizens.
		\end{itemize}
\end{frame}

\begin{frame}{Principles of Khana Ratsadon}
	\begin{block}{Core Principles}
		\begin{enumerate}
			\item Autonomy
			\item Security
			\item Economy
			\item Equality
			\item Ethics
			\item Education
		\end{enumerate}
	\end{block}
\end{frame}

%========================
% Section: Constitutional Court
%========================
\section{Constitutional Court}

\begin{frame}{Part III: Constitution}
	\begin{block}{Key Function}
		Constitution is the \textcolor{red}{primary function}, where the court examines laws and government
acts to determine if they are consistent with the constitution.
	\end{block}	
\end{frame}

\begin{frame}{Purposes of Constitution}
	\begin{block}{1) Protection of Rights and Liberties}
		The court plays a vital role in protecting the fundamental rights and freedoms
garanteed by the constitution to the people.
	\end{block}
	\begin{block}{2) Uphold the Rule of Law}
	By ensuring that political organs and institutions adhere to the constitution, the court
helps maintain the supremacy of constitutional law.
	\end{block}
	\begin{block}{3) Adjudicate Constitutional Disputes}
	The court resolves problematic issues of constitutionality that arise from legislation,
regulations, or government actions.
	\end{block}	
\end{frame}


\begin{frame}{Constitutional Court}
	\begin{exampleblock}{What is Constitutional Court?}
	A \textcolor{red}{specialised judicial body} with the exclusive power to interpret the constitution,
review the constitutionality of laws and government actions, and safeguard the rights and
liberties of citizens.
	\end{block}
	\begin{alertblock}{Role of Constitutional Court}
		Unlike courts of general jurisdiction, a constitutional court’s primary role is to ensure
that all legislative acts, royal decrees, and administrative actions comply with the
supreme constitutional law of the land.
	\end{block}
\end{frame}

\begin{frame}{A Specialized Judicial Body?}
The constitutional court isn’t mandatory for democratic countries.
		\begin{itemize}
			\item The Constitutional courts are found in more than 80 countries, particularly in civil law
country.
			\item The concept of a specialized constitutional court, following the model of Hans
Kalsen, became significant after World War II and was influential in the democratic
reconstruction of countries like Germany, and Italy.
		\end{itemize}
\end{frame}

\begin{frame}{Constitutional Court of Thailand}
	The concept of a constitutional court in Thailand was \textcolor{red}{modeled after European}
examples, such as \textit{the Constitutional Courts of Austria} and \textit{Germany}, and was first
established by the 1997 Constitution. This, then, re–established undr the 2007 and 2017
Constitutions, though with changing structures and powers.
\end{frame}

\begin{frame}{Constitutional Court of Thailand according to 2017 Constitution}
	A specialized judicial body responsible for reviewing the constitutionality of laws and
bills, ensuring the supremacy of the constitution, and adjudicating questions regarding the
powers and duties of government bodies.
	\begin{itemize}
		\item It consists of a President and eight Justices who are approved by the Senate and
appointed by the King.
		\item The court’s powers include ruling on draft amendments to the Constitution and
ensuring that governmental actions and laws align with the Constitution, which also
protects human rights and liberties.
	\end{itemize}
\end{frame}
%========================
% Thank You
%========================
\begin{frame}{Thank You}
  Questions and discussion are not welcome!  
  \vspace{1em}
  Contact: \texttt{punnawatcont@gmail.com}
\end{frame}

\end{document}
